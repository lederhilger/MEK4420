Vi husker følgende uttrykk for bevegelseslikningen for et fjærmassesystem.
\[
X = m\deet^2 \xi = -k\xi, \qquad \xi(t) = \hat{\xi}\sin{(\omega t)}, \qquad \omega = \sqrt{\sfrac{k}{m}}
\]
Vi går ikke veldig i dybden på elastisitet, men vi noterer oss at \textsc{Hooke}s lov også kan formuleres med hensyn på strekkspenning $\mathfrak{P}$ og tøyning $\epsilon$.\footnote{\cite{sokolnikoff1956mathematical} \textsc{Sokolnikoff}, p.68}
\[
\mathfrak{P} = \epsilon\youngsmodulus, \qquad L\epsilon = \xi,
\]
hvor $\youngsmodulus$ er \textsc{Young}s modul, en empirisk verdi, og spenningen er definert til å være kraften per flateareal.
Vi har altså at den effektive fjærstivheten kan gis ved
\[
k = \frac{\youngsmodulus A}{L}.
\]
