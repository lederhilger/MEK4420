Massen vi ser på i forankringssystemet er $m + m_{11}$ i jag, og $m + m_{22}$ i hiv.
I jag tilnærmer vi den adderte massen å være fortrengt vannmasse.
I hiv kan vi regne den ut.
Ved å bruke oblate sfæroidiske koordinater,\footnote{\cite{kennard1967irrotational} \textsc{Kennard}, pp.356--361} kan vi finne den kinetiske energien til en sirkulær disk med radius $a$ som beveger seg perpendikulært til flaten,
\[
T = \frac{k^{\prime} m_{22} U^2}{2} = \frac{4\varrho a^3 U^2}{3}, \qquad k^{\prime} = \frac{2}{\pi}.
\]
For sparbøyen, vil denne platen tilsvare bunnflaten, hvor baksiden ikke vil kunne opptas av fluid, som betyr at vi må halvere den adderte massen,
\[
m_{22} = \frac{2\pi\varrho {r_2}^3}{3}.
\]
