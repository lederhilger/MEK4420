Sirkelen finner vi enkelt potensialet på, så her kan vi kjapt sammenlikne utregnet potensial med det analytiske.
$L^2$-feilen blir 5\% allerede ved $N = 100$.

Vi har at
\[
        \phi = \frac{a^2 U \cos{\theta}}{r}, \qquad q^2 = \frac{a^4 U^2}{r^4},
\]
og finner den kinetiske energien til fluidet
\[
        T_{\mathrm{fluid}} = \frac{\varrho}{2} \int_{a}^{\infty} \int_{0}^{2\pi} q^2 r \,\dee \theta\,\dee r = \frac{\pi \varrho a^2 U^2}{2}.
\]
Vi bruker innsikten at $T_{\mathrm{total}} = \sfrac{1}{2} (m+\varrho kA) U^2$, altså at det kreves mer energi å forflytte et legeme i et fluid.

\vspace{1em}
$k$ er \emph{treghetskoeffisienten}.
Addert masse $m_{11} = \varrho kA$ er altså ikke bare avhengig av omsluttet areal, men også form i forhold til forflytning.
