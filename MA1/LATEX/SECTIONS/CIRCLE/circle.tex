Let $a$ be the radius of the circle.
We then have that its potential as a result of horizontal translation is given by
\[
w = a^2 U\ezh^{-1} = \Phi + i\Psi, \quad \ezh = x + iz.
\]
We see that
\[
\phi_1 = \Re{\left( a^2 U\ezh^{-1} \right)} = \frac{a^2 U \cos{\theta}}{r^2};
\]
\[
\phi_2 = \Re{\left( a^2 U i \ezh^{-1} \right)} = \frac{a^2 U \sin{\theta}}{r^2}.
\]
By calculating the gradient of $\phi$ and transforming into polar coordinates, we find that
\[
q_{r} = \frac{a^2 U \cos{\theta}}{r^2}, \quad q_{\theta} = \frac{a^2 U \sin{\theta}}{r^2}, \quad q^2 = \frac{a^4 U^2}{r^4}.
\]
We may now calculate the kinetic energy of the fluid by integrating over the entire fluid domain $\Omega$,
\[
T_{\mathrm{fluid}} = \frac{\varrho}{2} \int_{0}^{2\pi}\int_{a}^{\infty} q^2 r \,\dee r \dee \theta = \frac{\pi \varrho a^2 U^2}{2}.
\]
Rotating the cylinder ought not induce drift in the fluid, as there is no mechanisms by which the fluid should be compelled to move from the circle turning.
Because of the rotational symmetry of the circle, we must have that
\[
\addedmass = [m_{ij}]  = \begin{bmatrix}
           \pi \varrho a^2 & 0 & 0\\
           0 & \pi \varrho a^2 & 0\\
           0 & 0 & 0
\end{bmatrix}.
\]

We discretize the circle by defining
\[
\thetatt = \mathtt{linspace}(0, 2\pi, \mathtt{N}+1), \quad \xtt = a\cos{\thetatt} + ia\sin{\thetatt}.
\]
