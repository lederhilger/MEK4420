The arithmetic-geometric mean provides a swift way to calculate elliptic integrals numerically.
Its formulation seems rather arbitrary, but we will quickly see that there is a foundational connection between them.
Further reading is highly encouraged, in particular \emph{Pi and the AGM}\footnote{\cite{borwein1987pi} \textsc{Borwein} \& \textsc{Borwein}} by the \textsc{Borwein} brothers.
Following the literature, we define the arithmetic geometric mean $M$ of $a$ and $b$ through the iterative process
\[
a_{n} = \frac{a_{n-1} + b_{n-1}}{2}, \qquad b_{n} = \sqrt{a_{n-1} b_{n-1}}.
\]
This process is maintained until $c_{n} = \sfrac{1}{2}(a_{n-1} - b_{n-1}) < \epsilon$, where $\epsilon$ is the desired accuracy.
We have assumed that $0 < b_0 \leq a_0$, and we may observe that it is always true that $b_{n-1} \leq b_n \leq a_n \leq a_{n-1}$, so that the values have a common limit.
This limit is the arithmetic-geometric mean:
\[
M(a, b) = \lim_{n \to \infty} a_n = \lim_{n \to \infty} b_n.
\]
It is evident that from this definition, the starting the algorithm at some later iteration must result in the same function.
In other words,
\[
M(a, b) = M{\left( \frac{a + b}{2}, \sqrt{ab} \right)}.
\]
The arithmetic-geometric mean is also obviously homogeneous, meaning $M(\lambda a, \lambda b) = \lambda M(a,b)$, where $\lambda > 0$.
It may be shown that\footnote{\cite{borwein1987pi} \textsc{Borwein} \& \textsc{Borwein}, pp.5--7}
\[
\frac{\pi}{2 M(1, \ezh)} = \int_{0}^{\sfrac{\pi}{2}} \frac{1}{\sqrt{1 - \big( 1 - \ezh^2\big)\sin^2{\theta}}} \,\dee\theta.
\]
A full proof is not appropriately reproduced in this text, so the reader is referred to the \textsc{Borwein} brothers' book.
The relation to elliptic integrals is now quite apparent, in particular
\[
\ellK(k) = \frac{\pi}{2 M(1, k^{\prime})}, \quad \frac{\ellE(k)}{\ellK(k)} = 1 - \sum_{n} 2^{n-1} {c_n}^2.
\]
These are the \texttt{K} and \texttt{E} methods implemented in the \texttt{Jacobi} class of \texttt{jacobi.py}.
It is known that these algorithms only converge
