To study ellipses in the complex plane, we consider the \textsc{\v{Z}ukovsky} transform\footnote{\cite{kennard1967irrotational} \textsc{Kennard}, p.173}
\begin{equation}\label{eq:zukovsky_transform}
  \ezh = \ezh^{\prime} + \frac{c^2}{\ezh^{\prime}}, \qquad \ezh^{\prime} = x^{\prime} + iz^{\prime},
\end{equation}
so that $x^{\prime} = r\cos{\theta}$, $z^{\prime} = r\sin{\theta}$.
Here the prime is no more than a decorator to differentiate the preimage of the \textsc{\v{Z}ukovsky} transform from the complex numbers we will use.
We now have for the complex variable $\ezh = x + iz$, whereupon using \textsc{Euler}'s formula,\footnote{\cite{kennard1967irrotational} \textsc{Kennard}, pp.190--192}
\[
x = \left( r + \frac{c^2}{r} \right)\cos{\theta}, \qquad z =  \left( r - \frac{c^2}{r} \right)\sin{\theta}.
\]
This is indeed an ellipse, as upon squaring both terms and adding them, we get that
\[
\frac{x^2}{a^2} + \frac{z^2}{b^2} = 1, \quad a = r + \frac{c^2}{r}, \quad b = \left\vert r - \frac{c^2}{r} \right\vert.
\]
We call $a$ and  $b$ the major and minor semiaxes, corresponding to the longest and shortest semidiameters of the ellipse, respectively.
The parameter $c = \sqrt{a^2 - b^2}$ is called the focus.
Introducing now the elliptic complex variable $\zeta$, defined by
\[
\ezh^{\prime} = c e^{\zeta}, \qquad \zeta = \xi + i\eta,
\]
where $\xi = \ln{(\sfrac{r}{c})}$ and $\eta = \theta$, we may express the complex number $\ezh$ in terms of this elliptic complex variable such that
\begin{equation}\label{eq:elliptic_ezh}
\ezh = c\cosh{\zeta} = c \big( \cosh{\xi} \cos{\eta} + i \sinh{\xi} \sin{\eta} \big).
\end{equation}
We now clearly see that cricles in $\ezh^{\prime}$ are ellipses in $\ezh$.
Lines of constant $\xi$ constitute confocal ellipses, parametrized with $\eta$, whilst lines of constant $\eta$ are confocal hyperbolas in $\xi$.
We may eliminate either $\xi$ or $\eta$ in equation \eqref{eq:elliptic_ezh} to find that the major and minor semiaxes are respectively given by
\begin{equation}\label{eq:aprime_and_bprime}
a^{\prime} = c\cosh{\xi}, \qquad b^{\prime} = c\absl{\sinh{\xi}}.
\end{equation}
Combining now equations \eqref{eq:elliptic_ezh} and \eqref{eq:aprime_and_bprime},  we find that we may write
\begin{equation}\label{eq:elliptic_xi}
\xi = \ln{\left( \frac{a^{\prime} + b^{\prime}}{c} \right)}, \qquad \xi \geq 0.
\end{equation}
We will use this convention, that $\xi$ be strictly non-negative, though more discussion on this is to be found in \textsc{Kennard}'s work.\footnote{\cite{kennard1967irrotational} \textsc{Kennard} sect.82, pp.192--196}
It is then apparent that the potential for a circle, described in equation \eqref{eq:potential_circle}, may be transformed to give
\begin{equation}\label{eq:potential_ellipse}
w = U\sqrt{\frac{a + b}{a - b}} (b \cos{\alpha} + i a \sin{\alpha}) e^{-\zeta},
\end{equation}
where $\alpha$ is the inclination of the ellipse relative to the direction of motion.
That is, an angle $\alpha = 0$ corresponds to motion along $x$, and $\alpha = \sfrac{\pi}{2}$ corresponds to motion along $z$.
Clearly, by virtue of the above discussion, the real part of the potential is given by
\[
\Phi(\zeta; \alpha) = U \sqrt{\frac{a + b}{a - b}} (b\cos{\alpha} \cos{\eta} + a \sin{\alpha} \sin{\eta}) e^{-\xi},
\]
so that $\phi_1 (\zeta) = \Phi(\zeta; 0)$ and $\phi_2 (\zeta) = \Phi(\zeta; \sfrac{\pi}{2})$.
