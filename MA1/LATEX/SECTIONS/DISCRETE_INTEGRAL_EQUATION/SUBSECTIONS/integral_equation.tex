Since the fluid is ideal, we may reduce the incompressibility condition to the \textsc{Laplace} equation,
\[
\nabla^2\phi = 0, \qquad \text{in } \Omega.
\]
It may be shown with the product rule that for any harmonic functions $\phi,\psi \in \Omega$,
\[
\nabla\cdot \big( \phi\nabla\psi - \psi\nabla\phi \big) \equiv 0,
\]
which upon being integrated over $\Omega$ and applying \textsc{Gauss}' divergence theorem, yields
\[
\int_{\partial\Omega} \big( \phi\nabla\psi - \psi\nabla\phi \big)\cdot\nhat \,\dee S = 0.
\]
Since we only need the values of the potential on the contour $\partial\Omega$ to calculate the added mass, we look to \textsc{Green} functions, which are defined through the property that they satisfy some differential operator except at some point $\bzhe \in \Othal$, where $\Othal = \Omega \cup \partial\Omega$, such that
\[
\nabla^2 \green = \delta(\xvec - \bzhe), \qquad \xvec\in\Othal.
\]
In other words, the \textsc{Green} function is almost harmonic, and we expect the above integral should hold except at the pole.
The \textsc{Green} function for the \textsc{Laplace} operator is the natural logarithm,
\[
\green(\xvec) = \ln{r}, \qquad r = \absl{\xvec - \bzhe},
\]
which has a pole at $\bzhe$.
Now, by placing $\bzhe\in\partial\Omega$, we find by the \textsc{Cauchy} principal value,\footnote{\cite{lavrentev1967methoden} \textsc{Lavrentev} \& \textsc{\v{S}abat}, pp.331--332} that
\[
-\pi\phi(\bzhe) + \mathrm{pv}\int_{\partial\Omega} \phi \partial_{\nhat}\ln{r} \,\dee S = \mathrm{pv} \int_{\partial\Omega} \partial_{\nhat}\phi \ln{r} \,\dee S.
\]
An outline of calculating the principal value is found in the lecture notes from January 21\textsuperscript{st}.
We drop the principal value notation for brevity.
