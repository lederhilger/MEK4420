The problem at hand is to model the motion of a moored floating spar turbine through steady state force considerations, with tower specifications provided by the project supervisor Tor Anders \textsc{Nygaard}.
Of the major obstacles with floating offshore wind turbines, the cost of materials is the greatest.
We should therefore explore alternatives to the steel chains being used today, perhaps the two most relevant being nylon and polyester.
These materials do of course have very nonlinear unsteady behavior in strain, but modeling this is beyond the scope of the present work---we will strictly be interested in the tower dynamics, not that of the ropes.

The methodology follows from that outlined in \textsc{Lee} \emph{et al.},\footnote{\cite{lee2024design} \textsc{Lee} \emph{et al.} (2024)} where, among other criteria, the necessity of period of above \SI{40}{\second} is outlined.
Although no derivations are included in the cited paper, the expressions for the spring stiffnesses derived below do concur with those found there.
