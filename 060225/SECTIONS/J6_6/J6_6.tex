It is perhaps appropriate to provide a concrete example before divulging in the general position of the radiation problem.
We imagine some geometry described by the boundary $S_{\mathrm{B}}$ bobbing in water with a heave motion $\xi_2(t) = \Re{\big({\hat{\xi}}_2 \exp{(i\omega t)}\big)}$.
The body of water may be bounded by some bottom bathymetry.
It is the effect of radiation we are interested in---the waves generated by the motion of the geometry.
\begin{figure}[H]
    \centering
    \begin{tikzpicture}
    \begin{scope}
        \def\imagelength{3}
        \def\bodylength{1}
        \def\bedpos{-1}\def\beddepth{.2}
        \draw[latex-latex] (0,0)--(0,.5) node[right]{$\xi_2$};
        \fill[pattern = north west lines] (-\imagelength, \bedpos-\beddepth)--(-\imagelength, \bedpos)--(\imagelength, \bedpos)--(\imagelength, \bedpos-\beddepth);
        \draw (-\imagelength, \bedpos)--(\imagelength, \bedpos);
        \draw plot [domain = -\imagelength:-{\bodylength/2}, samples = 200] (\x, {-.1*sin(\imagelength*pi*(\x + \bodylength/2) r)});
        \draw plot [domain = {\bodylength/2}:\imagelength, samples = 200] (\x, {.1*sin(\imagelength*pi*(\x - \bodylength/2) r)});
        \draw[thick] (-\bodylength/2, .2)--(-\bodylength/2, -.1)--(\bodylength/2, -.1)--(\bodylength/2, .2);
    \end{scope}
\end{tikzpicture}
\end{figure}
\noindent The kinematic boundary condition between the geometry and the water is given by
\[
    \nhat \cdot \uvec_{\mathrm{B}} = \nhat \cdot \nabla\Phi_{\mathrm{R}} \qquad \text{on } S_{\mathrm{B}},
\]
where $\uvec_{\mathrm{B}} = \deet\xi_2 \jhat$ is the velocity of the geometry.
As an ansatz, we say that the radiation potential is given by
\[
    \Phi_{\mathrm{R}}(\xvec, t) = \Re{\big( i \omega {\hat{\xi}}_2 \exp{(i\omega t)} \phi_2(\xvec) \big)}.
\]
Given $\nhat\cdot\uvec_{\mathrm{B}} = \omega {\hat{n}}_y \xi_2$, we get the reformulated kinematic boundary condition
\[
    \Re{\Big( i \omega {\hat{\xi}}_2 \exp{(i \omega t)} \big( {\hat{n}}_y - \partial_{\nhat} \phi_2 \big) \Big)} = 0 \qquad \text{on } S_{\mathrm{B}}.
\]
We recall the linearized dynamic boundary condition for surface waves,
\[
    \deet^2 \Phi_{\mathrm{R}} = - \gravity \deey \Phi_{\mathrm{R}} \qquad \text{at } y = 0.
\]
We have that $\deet^2 \Phi_{\mathrm{R}} = -\omega^2 \Phi_{\mathrm{R}}$, yielding the dynamic boundary condition
\[
    \Re{\Big( i \omega {\hat{\xi}}_2 \exp{(i \omega t)} \big( \gravity \deey\phi - \omega^2 \phi \big) \Big)} = 0 \quad \text{at } y = 0.
\]

\newpage
\noindent With this specific example in mind, we say that the general radiation potential is given as a superposition of the potential modes,
\[
    \Phi_{\mathrm{R}}(\xvec, t) = \Re \sum_{j} i \omega \xi_j \phi_j(\xvec) \exp{(i \omega t)}.
\]
The kinematic boundary condition at the body is
\[
    \partial_{\nhat} \phi_j = {\hat{n}}_j \qquad \text{on } S_{\mathrm{B}}.
\]
When $\bzhe$ is on $S_{\mathrm{B}}$, the integral equation is given by
\[
    \pi \phi(\bzhe) = \int_{S_{\mathrm{B}}} \phi_j \partial_{\nhat} \green - \green \partial_{\nhat} \phi_j \,\dee S,
\]
and when $\bzhe$ is in the fluid, the left hand side of the equation is of twice the magnitude.

\subsection{Damping}
We notice now that the potentials $\phi_j$ may indeed be complex, indicating that the added mass may only represent the real part of the added mass integral.
Recall that $\Phi_i = i\omega {\hat{\xi}}_i \phi_i e^{i\omega t}$.
Omitting the temporal exponential, we find that the pressure is given by \textsc{Bernoulli}'s equation $p_i = i \varrho \omega {\hat{\xi}}_i \phi_i$.
We write that the displacement amplitude velocity and acceleration are given by $i\omega {\hat{\xi}}_i$ and $-{\omega}^2 {\hat{\xi}}_i$, repsectively.
We then have that
\begin{equation*}
    \begin{aligned}
        F_{ij} (t) & = -\varrho {\omega}^2 {\hat{\xi}}_{i} \int_{S_{\mathrm{B}}} \phi_i {\hat{n}}_j \,\dee S\\
        & = -{\omega}^2 {\hat{\xi}}_i m_{ij} + i\omega {\hat{\xi}}_i r_{ij},
    \end{aligned}
\end{equation*}
where the last equality is an ansatz due to us expecting an imaginary part, extrapolating from \textsc{Newton}'s second law of motion.
We then say that the \emph{added mass} $\addedmass$, and \emph{radiation damping} $\bm{\mathfrak{r}}$ tensors are defined as follows.
\[
    \addedmass = \varrho \Re \int_{S_{\mathrm{B}}} \phi_i {\hat{n}}_j \,\dee S, \quad \bm{\mathfrak{r}} = -\omega\varrho \Im \int_{S_{\mathrm{B}}} \phi_i {\hat{n}}_j \,\dee S
\]

\subsection{Example: \emph{The work done by a geometry}}
We want to find the rate of work for a motor that sustains the heave in the illustration above.
In general the average rate of work is given by $\overline{\deet W} = -\overline{F_2 U_2}$, where the overline is the time average, introducing a factor of one half.
We have that $F_2 = -\omega^2 {\hat{\xi}}_2 m_{22} + i\omega {\hat{\xi}}_2 r_{22}$, and $U_2 = i \omega {\hat{\xi}}_2$.
Since the rate of work is a real value, we have that 
\[
    \overline{\deet W} = \frac{\omega^2 {{}{\hat{\xi}}_{2}}^{2} r_{22}}{2}.
\]
We recall the following approximations for the kinetic and potential energies from the \emph{Hydrodynamic Wave Theory} course, being
\[
    \frac{\varrho}{2} \overline{\int_{-h}^{\eta} \uvec\cdot\uvec \,\dee y} \sim \frac{\varrho \gravity \absl{A}^2}{4}, \quad \frac{\varrho}{2} \overline{\int_{-h}^{\eta} \varrho \gravity y \,\dee y} \sim \frac{\varrho \gravity \absl{A}^2}{4},
\]
respectively.
We also recall the approximation for the energy flux,
\[
    \overline{\int_{-h}^{\eta} p u \,\dee y} \sim E c_g, \qquad c_g = \frac{\gravity}{2\omega}
\]
where $E$ is the sum of potential and kinetic energies, and $c_g$ is the group velocity for deep water.
We have that 
\[
    {\absl{A}}^2 = {{}{\hat{\xi}}_2}^2 {\absl{{\hat{\eta}}_{2}^{\pm\infty}}}^2,
\]
requiring a superposition when considering the total energy flux.
The actual work done is the characteristic time multiplied by the temporal average, being equivalent to dividing by $\omega$ in this case.
The work put into the system must equal the energy flux in the waves, so we now get that
\[
    \frac{r_{22}}{\varrho \omega} = \frac{1}{2}\left( {\absl{A_{j}^{\infty}}}^2 + {\absl{A_{j}^{-\infty}}}^2 \right).
\]

\subsection{Example: \emph{Far field amplitudes of a symmetrical geometry}}
In the far field, when $\zhe\to\pm\infty$, we have that $\green = \Re{(f_1)} + i\Re{(f_2)}$.
Carrying out the calculation, we find that
\[
    \green(\xvec; \bzhe) = 2\pi i \exp{(\mp\ezh)} \quad \text{as } \zhe\to\pm\infty,
\]
where $\ezh = -i\kappa (\xvec - \bzhe^{\ast})$.
It follows that
\[
    \partial_x \green = \pm i\kappa \green, \quad \deey \green = \pm\kappa \green \quad \text{as } \zhe \to \pm\infty,
\]
yielding
\[
    2\pi\phi_j (\bzhe) = \pm\int_{S_{\mathrm{B}}} \left( i\kappa \phi_j \nhat^{\ast} - {\hat{n}}_j \right) \green \,\dee S.
\]
We say
\[
    A_{j}^{\pm\infty} = \mp \int_{S_{\mathrm{B}}} \left( \kappa\phi_j \nhat^{\ast} + i{\hat{n}}_j \right) e^{\kappa(y \pm ix)} \,\dee S,
\]
so that
\[
    \phi_j(\bzhe) = A_{j}^{\pm\infty} e^{-\kappa(\che \pm i\zhe)}, \qquad \text{as } \zhe \to \pm\infty.
\]
We recall the kinematic boundary condition, $\deet\eta = \deez\Phi_{\mathrm{R}}$, and the dynamic boundary condition, $\deet^2 \Phi_{\mathrm{R}} = -\gravity \deez\Phi_{\mathrm{R}}$.
Having already assumed a \textsc{Fourier} decomposition of the potential, we find that the far field surface elevation may be expressed as a superposition
\[
    \eta(\zhe, t) = \sum_{j} {\hat{\xi}}_j {\hat{\eta}}_{j}^{\pm\infty} \exp{\big( i(\omega t \mp \kappa \zhe) \big)},
\]
where ${\hat{\eta}}_{j}^{\pm\infty} = \kappa A_{j}^{\pm\infty}$.
Supposing a combined sway and heave motion of the geometry, the outgoing waves on the respective sides of the geometry, is given by
\[
    \eta(\zhe, t) = \kappa \left( {\hat{\xi}}_1 A_{1}^{\infty} + {\hat{\xi}}_2 A_{2}^{\infty} \right) e^{i(\omega t - \kappa \zhe)},
\]
\[
    \eta(\zhe, t) = \kappa \left( {\hat{\xi}}_1 A_{1}^{-\infty} + {\hat{\xi}}_2 A_{2}^{-\infty} \right) e^{i(\omega t + \kappa \zhe)},
\]
for $\zhe\to\infty$ and $\zhe\to-\infty$, respectively.

We see that we are now free to choose a combination of ${\hat{\xi}}_1$ and ${\hat{\xi}}_2$ such that the far field on the left-hand side disappears.
The normal vectors of a geometry symmetrical about the $y$-axis has the property that ${\hat{n}}_x(x,y) = -{\hat{n}}_x(-x,y)$, and ${\hat{n}}_y(x,y) = {\hat{n}}_y(-x,y)$.
Furthermore, $\phi_1(x,y) = -\phi_1(-x,y)$ and $\phi_2(x,y) = \phi_2(-x,y)$ on the geometry.
This yields the properties that $A_{1}^{\infty} = -A_{1}^{-\infty}$ and $A_{2}^{\infty} = A_{2}^{-\infty}$.
It follows that
\[
    \eta(\zhe, t) = 2\kappa {\hat{\xi}}_2 A_{2}^{\infty} e^{i(\omega t - \kappa \zhe)},
\]
whilst the far field in the negative direction is negligible.